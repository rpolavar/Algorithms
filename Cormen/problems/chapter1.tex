\documentclass{article}
%\usepackage{beamerthemeBerkeley}
% Use either the one above or the one below
%\usetheme{Berkeley}
\usepackage{hyperref}
\newcommand{\tab}{\hspace*{1 cm}}
\title{1-1 Comparison of Running Times}
\author{TM}
\date{\today}

\begin{document}
For each function $f(n)$ and time $t$ in the following table,
determine the largest size $n$ of a problem that can be solved in time
$t$, assuming that the algorithm to solve the problem takes $f(n)$
microseconds.
\begin{tabular}{c|c|c|c|c|c|c|c|}
\multicolumn{1}{c|}{} & 1      & 1      & 1    & 1   & 1    & 1   & 1\\
\multicolumn{1}{c|}{} & second & minute & hour & day & month & year
	& century \\
\hline $\lg n$ &  &  &  &  &  &  &  \\
\hline $\sqrt{n}$ &  &  &  &  &  &  &  \\
\hline $n$ &  &  &  &  &  &  &  \\
\hline $n \lg n$ &  &  &  &  &  &  &  \\
\hline $n^2$ &  &  &  &  &  &  &  \\
\hline $n^3$ &  &  &  &  &  &  &  \\
\hline $2^n$ &  &  &  &  &  &  &  \\
\hline $n!$ &  &  &  &  &  &   &  \\
\hline
\end{tabular}
a.~I decided to solve this problem programmatically.  I wrote a program called {\tt fs.py}.  I had to write an approximation function since some of the functions are not obviously invertable.  Check the program.
\end{document}